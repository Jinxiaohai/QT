\chapter{QBoxLayout}
\section{void QBoxLayout::addStretch(int stretch = 0)}
Adds a stretchable space (a QSpacerItem) with zero minimum size and stretch factor stretch to the end of this box layout.\\
函数的作用是在布局器中增加一个伸缩量,%
里面的参数表示QSpacerItem的个数,默认值为零,%
会将你放在layout中的空间压缩成默认的大小。\\
例如:一个layout布局器,里面有三个控件,一个放在最左边,%
一个放在最右边,最后一个放在layout的1/3处,这就可以通过addStretch去实现。\\
例子:用addStretch函数实现将nLayout的布局器的空白空间平均分配:
\begin{lstlisting}
  QHBoxLayout *buttonLayout=new QHBoxLayout;
  button1=new QPushButton();
  button2=new QPushButton();
  button3=new QPushButton();
  buttonLayout->addStretch(1);  //增加伸缩量
  buttonLayout->addWidget(button1);
  buttonLayout->addStretch(1);  //增加伸缩量
  buttonLayout->addWidget(button2);
  buttonLayout->addStretch(1);  //增加伸缩量
  buttonLayout->addWidget(button3);
  buttonLayout->addStretch(6);  //增加伸缩量
  //void QWidget::setContentsMargins(int left, int top, int right, int bottom)
  //Sets the margins around the contents of the widget to have the sizes left, top, right, and bottom.
  //The margins are used by the layout system, and may be used by subclasses to specify the area to draw in (e.g. excluding the frame).
  buttonLayout->setContentsMargins(0,0,0,0);
  setLayout(buttonLayout);
\end{lstlisting}
其中四个addStretch()函数用于在button按钮间增加伸缩量,%
伸缩量的比例为1:1:1:6,%
意思就是将button以外的空白地方按设定的比例等分为9份并按照设定的顺序放入buttonLayout布局器中。%
%%% Local Variables:
%%% mode: latex
%%% TeX-master: t
%%% End:
